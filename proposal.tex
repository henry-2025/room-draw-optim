\documentclass[12pt]{article}
\usepackage[margin=1in]{geometry}
\author{Henry Pick \small hpick@hmc.edu \\
\large Noah Nevens \small nnevens@hmc.edu}
\title{HMC Room Draw Optimization}
\begin{document}
    \maketitle
    \section*{Problem Statement}
    Room draw at Harvey Mudd College is the annual process in which returning students select their dorms for the coming academic year. The process requires large amounts of interpersonal communication to plan a draw strategies between groups that want to live with one another. There is also a ``mock draw'' session that lasts several hours to ensure that people's expectations and preferences are made public to other participants in the process. Finally, the real draw round is held to determine final room assignments.

    The process is time-intensive and it can also be stressful to those who have less social influence in the period of pre-draw communication. There are inevitably unforeseen changes that happen during real draw round that have cascading effects throughout the rest of the process. As political and frustrating as room draw might seem, it is inherently an optimization problem where agents (the participants) adjust their preferences until an equilibrium is reached. 

    Participants almost always have an implicit list of preferences before entering the draw. These include but are not limited to residence hall, room number, and the members of nearby rooms. All of this knowledge is accessible through communication but with the number of participants in the draw, it is impossible for every agent to act with full knowledge. Therefore, the equilibrium of draw cannot be a global optimum.
    
    There is no explicit quantitative utility measure for a set of decisions that are made. This leads part of the room draw optimization problem to be defining the utilities of certain outcomes. Most likely, we will use satisfaction of certain preferences as a heuristic for utility. We will have to use our own opinions and judgement to weight each of these satisfactions, however.

    This problem can be approached with varying degrees of complexity. At a base level, we will want to implement a matching process where, given a list of room preferences, a global optimum utility is achieved. Implementing this will fundamentally alter the room draw process, in which case we might want to simulate an selection round under the existing draw process and then compare utility results. If these two tasks can be accomplished with reasonable time to spare, we can focus on modeling more complex collaborative behaviors with nearby resident preferences.

    \section*{Preliminary Model}
    \subsection*{Room Preference Model}
    Room draw currently uses a round-based selection process, which assigns a priority number to each participant, determining the order in which their room selection takes place. Room draw rounds are then organized by seniority, meaning there are three class rounds. 


    \subsection*{Implementation}
    Fundamentally, what we are dealing with is known as the assignment problem, a special case of the transportation problem, which is in turn a special case of the minimum cost flow problem. A reduction converts this into a linear program. This then should qualify as a reasonable project undertaking for this course if we are working purely with LP. In a more efficient scheme, however, there are algorithms that are optimized for solving the transportation problem. 

    The Jonker-Volgelant algorithm, a variant of the Hungarian algorithm for assignment, runs in $O(n^3)$ time and has an implementation in the \texttt{scipy.optimize} library. If we want to work using a database for matrix data, this seems like a good candidate for optimization tooling.
\end{document}